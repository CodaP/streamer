%% Version 4.3.2, 25 August 2014
%
%%%%%%%%%%%%%%%%%%%%%%%%%%%%%%%%%%%%%%%%%%%%%%%%%%%%%%%%%%%%%%%%%%%%%%
% Template.tex --  LaTeX-based template for submissions to the 
% American Meteorological Society
%
% Template developed by Amy Hendrickson, 2013, TeXnology Inc., 
% amyh@texnology.com, http://www.texnology.com
% following earlier work by Brian Papa, American Meteorological Society
%
% Email questions to latex@ametsoc.org.
%
%%%%%%%%%%%%%%%%%%%%%%%%%%%%%%%%%%%%%%%%%%%%%%%%%%%%%%%%%%%%%%%%%%%%%
% PREAMBLE
%%%%%%%%%%%%%%%%%%%%%%%%%%%%%%%%%%%%%%%%%%%%%%%%%%%%%%%%%%%%%%%%%%%%%

%% Start with one of the following:
% DOUBLE-SPACED VERSION FOR SUBMISSION TO THE AMS
%\documentclass{ametsoc}

% TWO-COLUMN JOURNAL PAGE LAYOUT---FOR AUTHOR USE ONLY
\documentclass[twocol]{ametsoc}

\usepackage{booktabs}
\usepackage{hyperref}
\usepackage{gensymb}

%%%%%%%%%%%%%%%%%%%%%%%%%%%%%%%%
%%% To be entered only if twocol option is used

\journal{jamc}

%  Please choose a journal abbreviation to use above from the following list:
% 
%   jamc     (Journal of Applied Meteorology and Climatology)
%   jtech     (Journal of Atmospheric and Oceanic Technology)
%   jhm      (Journal of Hydrometeorology)
%   jpo     (Journal of Physical Oceanography)
%   jas      (Journal of Atmospheric Sciences)	
%   jcli      (Journal of Climate)
%   mwr      (Monthly Weather Review)
%   wcas      (Weather, Climate, and Society)
%   waf       (Weather and Forecasting)
%   bams (Bulletin of the American Meteorological Society)
%   ei    (Earth Interactions)

%%%%%%%%%%%%%%%%%%%%%%%%%%%%%%%%
%Citations should be of the form ``author year''  not ``author, year''
\bibpunct{(}{)}{;}{a}{}{,}

%%%%%%%%%%%%%%%%%%%%%%%%%%%%%%%%

%%% To be entered by author:

%% May use \\ to break lines in title:

\title{Streamer Project}

%%% Enter authors' names, as you see in this example:
%%% Use \correspondingauthor{} and \thanks{Current Affiliation:...}
%%% immediately following the appropriate author.
%%%
%%% Note that the \correspondingauthor{} command is NECESSARY.
%%% The \thanks{} commands are OPTIONAL.

    %\authors{Author One\correspondingauthor{Author One, 
    % American Meteorological Society, 
    % 45 Beacon St., Boston, MA 02108.}
% and Author Two\thanks{Current affiliation: American Meteorological Society, 
    % 45 Beacon St., Boston, MA 02108.}}

\authors{Coda Phillips\correspondingauthor{Department of Atmospheric and Oceanic Science., UW-Madison, 1225 W Dayton St., Madison, WI}}

%% Follow this form:
    % \affiliation{American Meteorological Society, 
    % Boston, Massachusetts.}

\affiliation{}

%% Follow this form:
    %\email{latex@ametsoc.org}

\email{coda.phillips@wisc.edu}

%% If appropriate, add additional authors, different affiliations:
    %\extraauthor{Extra Author}
    %\extraaffil{Affiliation, City, State/Province, Country}

%\extraauthor{}
%\extraaffil{}

%% May repeat for a additional authors/affiliations:

%\extraauthor{}
%\extraaffil{}

\newcommand{\FU}{Wm\textsuperscript{-2}}

%%%%%%%%%%%%%%%%%%%%%%%%%%%%%%%%%%%%%%%%%%%%%%%%%%%%%%%%%%%%%%%%%%%%%
% ABSTRACT
%
% Enter your abstract here
% Abstracts should not exceed 250 words in length!
%
% For BAMS authors only: If your article requires a Capsule Summary, please place the capsule text at the end of your abstract
% and identify it as the capsule. Example: This is the end of the abstract. (Capsule Summary) This is the capsule summary. 

\abstract{Using the Streamer radiative-transfer model we were able to simulate radiative fluxes in a 25-level atmosphere for 129 spectral bands spanning $\lambda = 300nm$ to $100\mu m$.
Two climates were selected for testing: a tropical and an Arctic. A baseline radiation balance was established with the two standard profiles.
Then, five experiments were conducted on each of the climates, permuting the concentrations of water vapor, carbon dioxide, ozone, and adding two types of clouds.
Increasing the water vapor fraction did not result in any significant radiative forcing.
After doubling CO\textsubscript{2} concentration, there was a small warming effect on the total radiation balance.
The radiative forcing was found to agree with predictions from other literature.
The effect of ozone depletion was a dramatic increase the amount of UVB radiation impinging the surface.
Experiments with adding clouds resulted in major differences in net radiation balance from the baseline and a dependency on cloud height was established.}

\begin{document}

%% Necessary!
\maketitle


%%%%%%%%%%%%%%%%%%%%%%%%%%%%%%%%%%%%%%%%%%%%%%%%%%%%%%%%%%%%%%%%%%%%%
% MAIN BODY OF PAPER
%%%%%%%%%%%%%%%%%%%%%%%%%%%%%%%%%%%%%%%%%%%%%%%%%%%%%%%%%%%%%%%%%%%%%
%

%% In all cases, if there is only one entry of this type within
%% the higher level heading, use the star form: 
%%
\section{Introduction}
Streamer is a forward radiative-transfer model (RTM), designed to input atmospheric profile information and output radiances. Any of the seven standard atmospheric profiles can be modified to accommodate deviations in the concentrations of water vapor, oxygen, carbon dioxide, and ozone. With this flexibility, Streamer can be a powerful tool for studying the radiative forcing effects of these variables independently of one another. Likewise, climate predictions rely on knowledge on the global and local radiation balance and can be improved with a computational model. In our experiments we will focus on water vapor, ozone, and carbon dioxide \citep{Key:1998}.

Radiative forcing due to atmospheric constituent gases is easily computed because the absorption profiles and concentrations of these gases are well understood and simple to add to the RTM. Nevertheless, the effects of these gases are relevant to better understanding challenges like anthropogenic climate change and ozone depletion. Similarly, the radiative forcing of clouds can be researched through Streamer to help answer questions like the effect of man-made cirrus clouds on diurnal temperature range. We will limit our research to a couple cloud experiments, though much more is left to explore in the number of clouds, optical depth, phase, altitude, etc.
\section{Methods}
Streamer operates on a single text input file containing the parameters needed to run. Noteable settings include selection of spectral bands, atmospheric profile, solar zenith angle, constituent gas fractions, and cloud parameters. Experiments will be performed by adjusting these values individually and examining the simulation output. The output will typically contain flux densities for a configurable band at each of 25 heights. We will be most concerned with flux densities at the 100 km top-of-atmosphere and the surface. It is possible to compute spectral flux densities at each level by computing flux at a single band, then iterating over all bands and dividing by the bandwidth to approximate spectral flux. We will also divide total flux densities into two bands: longwave and shortwave.

The Python script that wraps Streamer and the exact runtime settings used in this article are open-source and available online\footnote{\url{https://github.com/CodaP/streamer}}.

There are two baseline atmospheric profiles: Arctic and tropical. Both profiles contain no clouds and rely on an oblique solar angle to result in comparable shortwave and longwave fluxes. The solar zenith angle is 71\degree{} in the tropics and 84\degree{} in the Arctic.

Fluxes will be computed for each baseline case. For the first experiment, water vapor concentrations will be increased by 10\% everywhere. Next, carbon dioxide concentrations will be doubled to determine the radiative forcing from emissions. Then, ozone concentrations will be reduced by 50\% to simulate ozone depletion. Finally, two experiments will be conducted by adding clouds to the radiative transfer model. One cloud will be a water-phase cloud located at 1 km height and with an optical depth of 10. The other cloud will have the same optical depth, but be ice-phase at 10 km height.

\section{Results}

\subsection*{Baseline Tables}

It is clear from \autoref{tab:tsr} and \autoref{tab:asr} that much more shortwave radiation reaches the surface in the tropical model compared to the Arctic model. Though both the Arctic and tropical models show the surfaces to have around 25\% albedo, overall albedo looking down from the top-of-atmosphere is significantly higher in the Arctic, nearly 40\%. The tropical overall albedo is 28\%, not much greater than the surface albedo. The oblique solar zenith angle in the Arctic model parameters is likely the cause and adjusting the Arctic solar zenith did reduce albedo to equal the tropical albedo. Closer inspection of the atmospheric bi-directional reflection function (BDRF) used internally by Streamer could confirm or refute the theory that a non-Lambertian BDRF is being considered by the model. In addition, according to the insolation equation $F = S_0 cos \theta_s$ \citep[p.~51]{petty:2008} the 13\degree{} decrease in solar elevation causes the normal flux to reduce by 68\%, further distancing the tropical and Arctic shortwave models.

In the Arctic, the shortwave flux absorbed at the surface is about equal to the emitted longwave flux. This indicates that the surface is likely at thermodynamic equilibrium and will not experience heating or cooling from radiative transfer. If Stefan-Boltzmann’s law is applied, the Arctic surface has a brightness temperature of about -31\degree{C}, in agreement with typical Arctic surface temperatures. However, the tropical shortwave flux exceeds the losses from longwave emission, suggesting that the surface is heating. Again using Stefan-Boltzmann’s law, the surface brightness temperature is about 27 \degree{C}, a reasonable temperature for a tropical environment.

\begin{table}[h]
    \centering
    \caption{Tropical Shortwave Radiation}
    \label{tab:tsr}
    \begin{tabular}{lccc}
        \toprule
        \multicolumn{4}{c}{$\textbf{Tropical}$}\\
        \midrule
        & $F^\uparrow_{sw}$ & $F^\downarrow_{sw}$ & $F^\downarrow_{net}$\\
        \midrule
        TOA &    125.0 &  450.83 &  325.83 \\
\midrule
SFC &     70.9 &  304.04 &  233.15 \\

        \bottomrule
    \end{tabular}
\end{table}
\begin{table}[h]
    \centering
    \caption{Tropical Longwave Radiation}
    \label{tab:tlr}
    \begin{tabular}{lccc}
        \toprule
        \multicolumn{4}{c}{$\textbf{Tropical}$}\\
        \midrule
        & $F^\uparrow_{lw}$ & $F^\downarrow_{lw}$ & $F^\downarrow_{net}$\\
        \midrule
        TOA &   290.06 &    0.00 & -290.06 \\
\midrule
SFC &   458.22 &  395.98 &  -62.24 \\

        \bottomrule
    \end{tabular}

\end{table}
\begin{table}[h]
    \centering
    \caption{Tropical LW+SW Radiation}
    \label{tab:tcr}
    \begin{tabular}{lccc}
        \toprule
        \multicolumn{4}{c}{$\textbf{Tropical}$}\\
        \midrule
        & $F^\uparrow$ & $F^\downarrow$ & $F^\downarrow_{net}$\\
        \midrule
        TOA &     415.06 &       450.83 &       35.77 \\
\midrule
SFC &     529.12 &       700.02 &      170.91 \\

        \bottomrule
    \end{tabular}

\end{table}

\begin{table}[h]
    \centering
    \caption{Arctic Shortwave Radiation}
    \label{tab:asr}
    \begin{tabular}{lccc}
        \toprule
        \multicolumn{4}{c}{$\textbf{Arctic}$}\\
        \midrule
        & $F^\uparrow_{sw}$ & $F^\downarrow_{sw}$ & $F^\downarrow_{net}$\\
        \midrule
        TOA &  57.98 &  144.74 &  86.76 \\
\midrule
SFC &  21.15 &   81.99 &  60.84 \\

        \bottomrule
    \end{tabular}
\end{table}

\begin{table}[h]
    \centering
    \caption{Arctic Longwave Radiation}
    \label{tab:alr}
    \begin{tabular}{lccc}
        \toprule
        \multicolumn{4}{c}{$\textbf{Arctic}$}\\
        \midrule
        & $F^\uparrow_{lw}$ & $F^\downarrow_{lw}$ & $F^\downarrow_{net}$\\
        \midrule
        TOA &  170.99 &    0.00 & -170.99 \\
\midrule
SFC &  194.25 &  134.55 &  -59.70 \\

        \bottomrule
    \end{tabular}

\end{table}

\begin{table}[h]
    \centering
    \caption{Arctic LW+SW Radiation}
    \label{tab:acr}
    \begin{tabular}{lccc}
        \toprule
        \multicolumn{4}{c}{$\textbf{Arctic}$}\\
        \midrule
        & $F^\uparrow$ & $F^\downarrow$ & $F^\downarrow_{net}$\\
        \midrule
        TOA &     228.97 &       144.74 &      -84.23 \\
\midrule
SFC &     215.40 &       216.54 &        1.14 \\

        \bottomrule
    \end{tabular}

\end{table}

\subsection*{Radiative Forcing}
\subsubsection{10\% increase in water vapor}
The small increase in water vapor did not seem to have much effect on overall radiation balance or atmospheric heating rates.
The main difference from the baseline can be seen in \autoref{fig:arc_flux}(B) and \autoref{fig:trop_flux}(B).
Water vapor absorbs so strongly at some wavelengths that radiative transfer is effectively null. Increasing water vapor concentration may increase optical depth, but the depth is already so large that the difference in transmittance is negligible. The effect is saturated and even large increases in concentration will have little effect. In water vapor's strongest absorption bands above $20\mu m$ we see an increase in surface downwelling flux.
But, the effect is negligible compared to other experiments and the heatings rates shown in \autoref{fig:arc_heat}(B) and \autoref{fig:trop_heat}(B) don't seem to indicate any heating.

\subsubsection{200\% increase in CO2}
Again, there doesn't seem to be much immediate effect on the radiation balance in either location compared to clouds. The net increase in flux is 1 \FU~in the Arctic and 2.5 \FU~in the tropics.
Without feedback systems, doubling carbon dioxide has been calculated to result in a 3.7 \FU~radiative forcing, within range of our finding \citep{rahmstorf:2008}.
Obviously, the increased carbon dioxide causes radiative forcing in the absorption bands of carbon dioxide. From \autoref{fig:arc_flux}(C) and \autoref{fig:trop_flux}(C) it appears that these bands lie at about $15\mu m$. The top-of-atmosphere experiences a net increase in upwelling flux in center of the band, but it is flanked on either spectral side by an increase in net upwelling flux. Typically, carbon dioxide absorbs very strongly at $15\mu m$, and therefore the effect of increasing concentration is saturated. At the TOA emission in these bands is dominated by CO\textsubscript{2} in the stratosphere (apparent from \autoref{fig:arc_heat}(C) and \autoref{fig:trop_heat}(C)), where temperature increases with altitude. It is possible that increasing CO\textsubscript{2} concentration in the saturated region simply raises the level of peak emission in the stratosphere, therefore increasing emission. In the unsaturated edged of the absorbing band, surface emissions are being attenuated by the larger optical path. Ultimately, the dominant effect is the absorption of surface longwave flux and the heating rates of layers above saturation are diminished.

\subsubsection{50\% decrease in ozone}
After reducing ozone concentrations by 50\% we can see the expected response in the shortwave. Ozone in the upper atmosphere absorbs high-energy ultraviolet radiation, so it follows that reducing the amount of ozone would allow more ultraviolet radiation to reach the surface. Indeed we see an fivefold increase in UVB ($.289\mu m$) radiance at the tropical surface, which would be quite catastrophic for biological health. The Arctic model has such an extreme solar angle that an ozone fraction greater than 40\% effectively extinguishes all UVB radiation before it can reach the surface. Additionally, visible light levels also increase at the surface, translating to a marginal increase in absorbed shortwave radiation. In the longwave regime there is a noticeable increase in upwelling radiation around $10\mu m$ at the TOA and a corresponding small decrease in downwelling radiance at the surface. This can be explained by the more transparent atmosphere in these channels. From the surface you can see through to space and at the TOA you can see the warmer ground.

\subsubsection{1 km water cloud}
Introducing a low cloud results in an increased albedo and a barrier to longwave radiative transfer. In the shortwave regime there is a large increase in upwelling radiance at the top-of-atmosphere and an identical decrease in downwelling shortwave radiance at the surface. If the increase in upwelling shortwave at the TOA was less than the decrease in shortwave at the surface then it could be inferred that the cloud was absorbing. In this case the cloud does not seem to be a strong absorber and probably exhibits a high single-scatter albedo. Overall, the addition of the low cloud increases the albedo by 150\% and the shape of the upwelling spectrum matches that of the solar radiation. Therefore, it is likely that the cloud has a high, uniform reflectivity for most of the shortwave wavelengths.

These shortwave effects are shared by both climates, but the reactions differ with longwave forcing. In the tropics, we see an increase in downwelling longwave radiation at the surface and a decrease in the upwelling at the top-of-atmosphere. From this it can be supposed that the cloud almost perfectly absorbs thermal radiation. At the top-of-atmosphere, rather than thermal emission being transmitted from the surface, it is emitted from the slightly cooler cloud. At the surface, received thermal emission increased because the nearby warm cloud dominates emission once left to constituent gases. In contrast, the Arctic top-of-atmosphere experiences an increase in longwave flux, indicating that the cloud is warmer than the surface. This could be true given that the downwelling longwave flux is greater than the emitted flux from the surface. But, heating rates (shown in \autoref{fig:arc_heat}(E)) do not show increased heating directly above the cloud. A higher vertical resolution would be helpful to deduce exactly what is occurring here.

\subsubsection{10 km ice cloud}
The effects of adding an ice cloud at 10 km on albedo is largely the same as adding the water cloud at 1 km.
Top-of-atmosphere albedo is increased by a little more with the ice cloud, probably due to the removal of 9 km of slightly absorbing atmosphere.
Albedo is now 70\% in the Arctic and 63\% in the tropics.
The absorption of shortwave by constituent gases is greater in the tropics than the Arctic because the irradiance is greater in the tropics, as can be seen comparing \autoref{fig:trop_heat}(A) and \autoref{fig:arc_heat}(A).
The differences in the longwave fluxes are more pronounced.
Whereas the low cloud resulted in change in radiation balance, retaining 42-114 \FU~more flux, the opposing effects of reflecting and insulating are more matched when the cloud is at 10 km.
This is because the cloud is much colder at the higher altitude.
Contributions to the surface and TOA longwave are diminished according to Stefan-Boltzmann's law; $\frac{\delta F}{\delta T} = 4.6~W m^{-2}K^{-1}$ at 0\degree{C}, meaning that the change in emitted flux is very sensitive to temperature changes at atmospheric temperatures.
The combined effect was a decrease in 19 \FU~in of net flux the tropics, and an increase of 8 \FU~in the Arctic.
These values are significantly less than what was seen with the low cloud, but it is easier to see the heating effect in \autoref{fig:arc_heat}(F) and \autoref{fig:trop_heat}(F).

\begin{figure*}[h]
    \centering
    \includegraphics[width=\linewidth, clip, trim=3cm 0cm 3cm 0cm]{figures/arctic_heating.eps}
    \caption{
    \emph{(A)} The heating rates for the baseline Arctic model parameters at each of the 25 altitude levels. The rates are divided into shortwave, longwave, and net components.
    \emph{(B)} Here is the difference in heating rates between the baseline case and a simulation of a 10\% increase in water vapor concentrations everywhere. The effect is limited because water vapor is mostly located in the troposphere where the absorption effect is saturated.
    \emph{(C)} The heating rates for CO\textsubscript{2} are affected by doubling concentration because carbon dioxide is well-mixed and and the extinction coefficient decreases with pressure. Like water vapor, CO\textsubscript{2} absorption is saturated in the troposphere, but in the upper-atmosphere its partial pressure drops and optical depth decreases steadily. This demonstrates how the CO\textsubscript{2} can cause global warming while water vapor does not have the same effect. 
    \emph{(D)} By decreasing ozone concentration by 50\%, the heating from shortwave radiation has diminished in the upper-atmosphere and it cools. Conditions at the surface and troposphere remain the same because most radiation is still overwhelmingly absorbed due to the low solar elevation.
    \emph{(E)} Here the low cloud blocks thermal radiation from the surface from reaching above the cloud and reduces longwave heating.
    \emph{(F)} The effect from an ice cloud with the same optical depth is an increase in longwave heating below the cloud. The heating is partially countered by a dimming of shortwave below the cloud.}
    \label{fig:arc_heat}
\end{figure*}

\begin{figure*}[h]
    \centering
    \includegraphics[width=\linewidth, clip, trim=3cm 0cm 3cm 0cm]{figures/tropical_heating.eps}
        \caption{
    \emph{(A)} The heating rates for the baseline Tropical parameters at each of the 25 altitude levels. The rates are divided into shortwave, longwave, and net components.
    \emph{(B)} Here is the difference in heating rates between the baseline case and a simulation of a 10\% increase in water vapor concentrations everywhere. The effect is limited because water vapor is mostly located in the troposphere where the absorption effect is saturated.
    \emph{(C)} The heating rates for CO\textsubscript{2} are affected at higher altitudes by doubling concentration because carbon dioxide is well-mixed and the extinction coefficient decreases with pressure. Like water vapor, CO\textsubscript{2} absorption is saturated in the troposphere, but in the upper-atmosphere its partial pressure drops and optical depth decreases steadily. This demonstrates how the CO\textsubscript{2} can cause global warming while water vapor does not have the same effect. 
    \emph{(D)} By decreasing ozone concentration by 50\%, the heating from shortwave radiation has diminished in the upper-atmosphere and it cools.
    \emph{(E)} Here the low cloud blocks thermal radiation from the surface from reaching above the cloud and reduces longwave heating above the cloud and increases heating underneath.
    \emph{(F)} The effect from an ice cloud with the same optical depth is an increase in longwave heating below the cloud. The heating is partially countered by a dimming of shortwave below the cloud. Directly above the cloud, heating is reduced because less thermal flux is being emitted from the cooler cloud compared with that coming from the baseline surface.}
    \label{fig:trop_heat}
\end{figure*}

\begin{figure*}[h]
    \centering
    \includegraphics[width=\linewidth, clip, trim=3cm 0cm 3cm 0cm]{figures/arctic_flux.pdf}
    \caption{
    \emph{(A)} The baseline Arctic parameters produce these spectral fluxes.
    Net fluxes are calculated by subtracting upwelling radiance from downwelling radiance.
    \emph{(B)} The difference between a simulation of 110\% water vapor concentration and the baseline. Positive values for surface longwave flux indicate an increase in these radiances correlating with a forced increase in water vapor concentration. They occur in saturated and unsaturated absorption bands of water vapor.
    \emph{(C)} The negative values show a decrease in top-of-atmosphere (TOA) upwelling longwave flux, concentrated in the CO\textsubscript{2} absorption bands when carbon dioxide levels are doubled.
    \emph{(D)} If ozone is halved, increases are seen in surface shortwave impingement. There is also an increase in upwelling flux at the TOA in ozone's longwave absorption band.
    \emph{(E)} The addition of a water cloud at 1 km with significant optical depth reflects light away from the surface and back to the TOA. This is manifested by the reduction in downwelling shortwave radiance at the surface and increase in reflected shortwave at the TOA. The thermal flux detected at the surface is now supplied by a cloud much warmer than the baseline background. Similarly, the thermal flux detected at the TOA increases because it is now produced by a warmer object than the surface.
    \emph{(F)} If the cloud is moved to 10 km the temperature of the cloud decreases and the effects on thermal emission at the surface are weakened. The increase in thermal IR at the TOA has been reversed and upwelling flux has decreased because cloud is cooler than the baseline surface.
    }
    \label{fig:arc_flux}
\end{figure*}

\begin{figure*}[h]
    \centering
    \includegraphics[width=\linewidth, clip, trim=3cm 0cm 3cm 0cm]{figures/tropical_flux.pdf}
    \caption{
    \emph{(A)} The baseline tropical model parameters produce these spectral fluxes.
    Net fluxes are calculated by subtracting upwelling radiance from downwelling radiance.
    \emph{(B)} The difference between a simulation of 110\% water vapor concentration and the baseline. Positive values for surface longwave flux indicate an increase in these radiances correlating with a forced increase in water vapor concentration. They occur in saturated and unsaturated absorption bands of water vapor.
    \emph{(C)} The negative values show a decrease in TOA upwelling longwave flux, concentrated in the CO\textsubscript{2} absorption bands when carbon dioxide levels are doubled.
    \emph{(D)} If ozone is halved, increases are seen in surface shortwave impingement. There is also an increase in upwelling flux at the TOA in ozone's longwave absorption band.
    \emph{(E)} The addition of a water cloud at 1 km with significant optical depth reflects light away from the surface and back to the TOA. This is manifested by the reduction in downwelling shortwave radiance at the surface and increase in reflected shortwave at the TOA. The thermal flux detected at the surface is now supplied by a cloud much warmer than the baseline background. Similarly, the thermal flux detected at the TOA decreases because it is now produced by a cooler object than the surface.
    \emph{(F)} If the cloud is moved to 10 km the temperature of the cloud decreases and the effects on thermal emission at the surface are weakened, but the effects on the TOA have strengthened. A large amount of thermal flux normally escaping into space is being blocked by the cloud.
    }
    \label{fig:trop_flux}
\end{figure*}

\section{Conclusions}
By running simulations using Streamer the characteristic radiative forcing for many atmospheric parameters could be determined.
The radiative forcing from doubling CO\textsubscript{2} was able to be calculated and the result was plausible.
The effects of clouds on radiation balance could be quantified and a dependence of surface flux on cloud height was ascertained.
\par There is some potential for inaccuracy when using Streamer.
For example, in the baseline cases chosen for atmospheric measurement the solar elevation angle is very low for the tropical model and extremely low in the Arctic model.
In the user guide, Streamer warns against using solar zenith angles greater than 70 degrees, a limit exceeded in both cases.
The source of errors at such angles arise because Streamer does not simulate spherical geometry or refraction, phenomena only significant at grazing angles of incidence.
Another shortfall is the assumption of a plane-parallel atmosphere, which is often a useful and valid simplification.
But, at low angles the amount of atmosphere traversed horizontally by an incident ray makes it unlikely to be laterally homogeneous.
\par Some interesting experiments not conducted here could further investigate carbon dioxide forcing.
Adding clouds to the model before determining radiative forcing might change results and would be necessary to accurately model the real world.
It would also be interesting to run the Streamer model many times, cycling throughout the day.
Then, one could calculate the average energy difference accumulated over the course of the day. This could add insight to the forcing effects of carbon dioxide.
Another improvement would be to simulate heating and cooling.
At each layer the excess radiative flux could converted into an incremental heating, applied to the atmospheric profile, and fed back into the model. After many iterations, it is likely that the system will reach equilibrium.
These results might be more realistic for determining the climate sensitivity because the equilibrium temperature of the upper-atmosphere is important in determining the intensity of escaping thermal radiation. Unfortunately, in our experiments the atmospheric temperature profile was fixed. Of course, the proposed iterative model would not account for atmospheric dynamics like convection, but the results might still be of interest.
\clearpage
%\section{References}

%%%%%%%%%%%%%%%%%%%%%%%%%%%%%%%%%%%%%%%%%%%%%%%%%%%%%%%%%%%%%%%%%%%%%
% ACKNOWLEDGMENTS
%%%%%%%%%%%%%%%%%%%%%%%%%%%%%%%%%%%%%%%%%%%%%%%%%%%%%%%%%%%%%%%%%%%%%
%
%\acknowledgments
%Start acknowledgments here.

%%%%%%%%%%%%%%%%%%%%%%%%%%%%%%%%%%%%%%%%%%%%%%%%%%%%%%%%%%%%%%%%%%%%%
% APPENDIXES
%%%%%%%%%%%%%%%%%%%%%%%%%%%%%%%%%%%%%%%%%%%%%%%%%%%%%%%%%%%%%%%%%%%%%
%
% Use \appendix if there is only one appendix.
%\appendix

% Use \appendix[A], \appendix}[B], if you have multiple appendixes.
%\appendix[A]

%% Appendix title is necessary! For appendix title:
%\appendixtitle{}

%%% Appendix section numbering (note, skip \section and begin with \subsection)
% \subsection{First primary heading}

% \subsubsection{First secondary heading}

% \paragraph{First tertiary heading}

%% Important!
%\appendcaption{<appendix letter and number>}{<caption>} 
%must be used for figures and tables in appendixes, e.g.,
%
%\begin{figure}
%\noindent\includegraphics[width=19pc,angle=0]{figure01.pdf}\\
%\appendcaption{A1}{Caption here.}
%\end{figure}
%
% All appendix figures/tables should be placed in order AFTER the main figures/tables, i.e., tables, appendix tables, figures, appendix figures.
%
%%%%%%%%%%%%%%%%%%%%%%%%%%%%%%%%%%%%%%%%%%%%%%%%%%%%%%%%%%%%%%%%%%%%%
% REFERENCES
%%%%%%%%%%%%%%%%%%%%%%%%%%%%%%%%%%%%%%%%%%%%%%%%%%%%%%%%%%%%%%%%%%%%%
% Make your BibTeX bibliography by using these commands:
\bibliographystyle{ametsoc2014}
\bibliography{references}


%%%%%%%%%%%%%%%%%%%%%%%%%%%%%%%%%%%%%%%%%%%%%%%%%%%%%%%%%%%%%%%%%%%%%
% TABLES
%%%%%%%%%%%%%%%%%%%%%%%%%%%%%%%%%%%%%%%%%%%%%%%%%%%%%%%%%%%%%%%%%%%%%
%% Enter tables at the end of the document, before figures.
%%
%
%\begin{table}[t]
%\caption{This is a sample table caption and table layout.  Enter as many tables as
%  necessary at the end of your manuscript. Table from Lorenz (1963).}\label{t1}
%\begin{center}
%\begin{tabular}{ccccrrcrc}
%\hline\hline
%$N$ & $X$ & $Y$ & $Z$\\
%\hline
% 0000 & 0000 & 0010 & 0000 \\
% 0005 & 0004 & 0012 & 0000 \\
% 0010 & 0009 & 0020 & 0000 \\
% 0015 & 0016 & 0036 & 0002 \\
% 0020 & 0030 & 0066 & 0007 \\
% 0025 & 0054 & 0115 & 0024 \\
%\hline
%\end{tabular}
%\end{center}
%\end{table}

%%%%%%%%%%%%%%%%%%%%%%%%%%%%%%%%%%%%%%%%%%%%%%%%%%%%%%%%%%%%%%%%%%%%%
% FIGURES
%%%%%%%%%%%%%%%%%%%%%%%%%%%%%%%%%%%%%%%%%%%%%%%%%%%%%%%%%%%%%%%%%%%%%
%% Enter figures at the end of the document, after tables.
%%
%
%\begin{figure}[t]
%  \noindent\includegraphics[width=19pc,angle=0]{figure01.pdf}\\
%  \caption{Enter the caption for your figure here.  Repeat as
%  necessary for each of your figures. Figure from \protect\cite{Knutti2008}.}\label{f1}
%\end{figure}

\end{document}