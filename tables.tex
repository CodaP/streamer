\documentclass[10pt]{article}

\usepackage{booktabs} % For \toprule, \midrule and \bottomrule
\usepackage{siunitx} % Formats the units and values
\usepackage{pgfplotstable} % Generates table from .csv

% Setup siunitx:
\sisetup{
  round-mode          = places, % Rounds numbers
  round-precision     = 2, % to 2 places
}

\begin{document}

\begin{table}[h!]
  \begin{center}
    \caption{Tropical Shortwave Radiation}
    \label{table1}
    \pgfplotstabletypeset[
      multicolumn names, % allows to have multicolumn names
      col sep=comma, % the seperator in our .csv file
      display columns/0/.style={
		column name=$SW\uparrow$, % name of first column
		column type={S},string type
	  },  % use siunitx for formatting
      display columns/1/.style={
		column name=$SW\downarrow$,
		column type={S},string type
	  },
	  display columns/2/.style={
		column name=$Net$,
		column type={S},string type
	  },
      every head row/.style={
		before row={\toprule}, % have a rule at top
		after row={\midrule} % rule under units
	  },
	  every last row/.style={after row=\bottomrule}, % rule at bottom
    ]{csv/tropical_sw.csv} % filename/path to file
  \end{center}
\end{table}

\begin{table}[h!]
  \begin{center}
    \caption{Tropical Longwave Radiation}
    \label{table2}
    \pgfplotstabletypeset[
      multicolumn names, % allows to have multicolumn names
      col sep=comma, % the seperator in our .csv file
      display columns/0/.style={
		column name=$LW\uparrow$, % name of first column
		column type={S},string type
	  },  % use siunitx for formatting
      display columns/1/.style={
		column name=$LW\downarrow$,
		column type={S},string type
	  },
	  display columns/2/.style={
		column name=$Net$,
		column type={S},string type
	  },
      every head row/.style={
		before row={\toprule}, % have a rule at top
		after row={\midrule} % rule under units
	  },
	  every last row/.style={after row=\bottomrule}, % rule at bottom
    ]{csv/tropical_lw.csv} % filename/path to file
  \end{center}
\end{table}

\end{document}